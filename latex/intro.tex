\section{Introduction}

\label{sec:intro}

Bankruptcy prediction of a company is, not surprisingly, a topic that
has attracted a lot of research in the past decades by multiple disciplines~\cite{altman-bankruptcy-17,kumar-review-07,chen-bankruptcy-11,lee-bankruptcy-13,erdogan-bankruptcy-13,cho-bankruptcy-10,wang-bankruptcy-11,Altman-8,Ohlson-9,Begley-10,Lee-10a,Fernandez-11,Odom-13,Atiya-15,Wang-16}.
The main importance of such research is in bank lending and, more in general, bankruptcy prediction is closely related to financial stability.
In particular, default prediction is one of the most challenging activities for managing credit risk.
In fact, banks need to predict the possibility of default of a potential counterpart before they extend a loan.
An effective predictive system can lead to a sounder and profitable lending decisions leading to significant savings for the banks and the companies and, most importantly, to a stable financial banking system.
A stable and effective banking system is crucial for
financial stability and economic recovery as well highlighted by the last global financial crisis started in 2008 and the next European debt crisis.
The current economic situation characterized by the crisis generated by the Covid pandemic will make a good ability to predict default even more relevant in the next period.

%The magnitude of bankruptcy costs is a critical issue in terms of
%capital structure theories.
%According to Fabio Panetta, former general director of the Bank of Italy,
%referring to Italian loans,
%``The growth of the new deteriorated bank loans and the slowness of the
%judicial recovery procedures have determined a rapid increase in the
%stock of these assets, which in 2015 reached a peak of 200 billion,
%equal to 11 percent of total loans.''\footnote{Fabio Panetta, Chamber of Deputies, Rome, May 10, 2018.}

Several recent and advanced techniques for predicting bankruptcy have been developed  over the years. Statistical and Machine Learning techniques are the two broad categories used to predict bankruptcy ~\cite{altman-bankruptcy-17,kumar-review-07,Atiya-15,Wang-16}.

Statistical techniques include linear discriminant analysis (LDA), multi-discriminant analysis (MDA) [5], logistic regression (LR), etc., while Machine learning techniques (ML) include well-known algorithms such as Artificial neural networks (ANN), SVM [6], Decision trees [7], Random Forest and Boosting techniques. \\



The focus of this article is on the empirical approach, especially the use of tree-based Machine Learning techniques. We will demonstrate that some of the more sophisticated boosting techniques achieve the best results when applied to this problem. Furthermore, we focus even more on what are the motivation that lead to a certain prediction. The explainability of a model is fundamental when it is applied to financial problems and for this reason, we used a modern approach called SHAP values that will show for the entire model and even for a single prediction which are the attributes that influence more the outcome. 

Of course, despite the plethora of studies, predicting the failure of a company is a hard task, as demonstrated by the enormous increase in large corporate failures in the last decades. Most related research has focused on bankruptcy prediction, which takes place when the company officially has the status of being unable to pay its debts (see Section 3). However, companies often signal much earlier their financial problems towards the banking system by going in default. Informally speaking, a company enters into a default state if it has failed to meet its requirement to repay its loans to the banks and it is very probable that it will not be able to meet his financial commitments in the future (again, see Section 3). Entering into a default state is a strong signal of a company’s failure: typically banks do not finance a company into such a state and it is correlated with future bankruptcy.
We try to predict both bankruptcy and bank default using credit data in combination with balance sheet data. We will show that this combination of information together with the use of particular feature selection techniques and advanced boosting techniques will lead us to obtain very relevant results even in comparison with the best results in the literature.

In this paper, we use historic data for predicting whether a company will enter in default. We base our analysis on the use of three different dataset (see~\ref{subsec:dataset-description}). First, we use historic information from all the loans obtained by almost all the companies based in Italy. This information includes information on the companies credit dynamics in the past years, as well as past information on relations with banks and values of protections associated with loans. 

The first dataset is based on information from Italian Central Credit Register which represents a historical Italian database containing information on the banking behaviour of companies and is widely used both by banks to assess the creditworthiness of counterparties and by banking supervisory authorities.
The second dataset, on the other hand, contains information from a new credit data collection recently carried out by the ECB, together with the National Central Banks (NCBs) of the euro area. This survey is called AnaCredit and is conducted in Italy by the Bank of Italy which collects important credit information on Italian companies from Italian banks, partly overlapping with those of the Central Credit Register and partly completely new. 
In any case, the information collected in AnaCredit allows for an even greater degree of detail than that of the Central Credit Register.
We will see below that these new features will prove to be very important in default predictions.
Due to the very recent period of use of the data, it is currently not possible to combine credit information with information from company balance sheets data. Such data are typically used in the literature to predict corporate failure and can also significantly improve bank default predictions (see Aliaj et al.)\\
The third dataset that we use, in combination with the first, contains a set of balance sheet indicator related to the Italian firms.


Note that the dimensions and the information in our dataset are very
significant also in comparison those of past work ~\cite{altman-bankruptcy-17,Atiya-15}, allowing to obtain a very accurate
picture of the possibility to predict over various economic sectors.
In addition we use for the first time, to predict firm's bank default, a very recent source of credit information which will be widely used in the future for important Central bank uses by both the ECB and NCBs.




\paragraph{Contributions.} To summarize the contributions of our paper are:
\begin{enumerate}
\item We analyze two very large credit datasets (over $570K$ companies) with highly granular
data on the performance of each company over a period of one year. We combine credit data with balance sheet data and besides, for the first time, two relevant sources of credit information: Italian Central Credit Register and AnaCredit data.
\item We use these data to predict whether a company will default in the
next year, considering both bankruptcy and bank default prediction. We try to extract the most of the information using an accurate procedure of feature selection, in combination with some promising recent boosting techniques.
\item We provide a method to give a sound explanation of the model predictions.
\end{enumerate}

\paragraph{Roadmap.} 
In the remainder of this section we provide definitions and we
describe the problem that we solve. 
In Section~\ref{sec:related} we present some related work. 
In Section~\ref{sec:methods} we
describe all the techniques and algorithms used. In Section ~\ref{sec:approach} we describe our dataset and in
Section~\ref{sec:experiments} we present our results. We conclude in
Section~\ref{sec:conclusion}. \footnote{The views expressed in the article are those of the authors and do not involve the responsibility of the Bank of Italy.}\\\\\\


Most related research has focused on \emph{bankruptcy} prediction, which
takes place when the company officially has the status of being unable
to pay its debts (see Section~\ref{sec:problem}). However, companies
%to pay its debts. However, companies
often signal much earlier their financial problems towards the banking
system by going in \emph{default}. Informally speaking, a company enters
into a default state if it has failed to meet its requirement to repay
%its loans to the banks (see Section~\ref{sec:problem}). Entering into a
its loans to the banks and it is very probable that it will not be able to meet his financial commitments in the future
(again, see Section~\ref{sec:problem}). Entering into a
default state is a strong signal of a company's failure: typically banks
do not finance a company into such a state and it is correlated with
future bankruptcy.

Firms bankruptcy prediction and more generally creditworthiness assessment of the companies can be very important also in \emph{policy decisions}, such as for example the policies of assignment of public guarantee programs ~\cite{andini-19}.



\subsection{Firm-Default--Prediction Problem}
\label{sec:problem}


There are many technical terms used to characterize debtors who are in
financial problems: illiquidity, insolvency, default, bankruptcy, and so
on. Most of the past research on prediction of failures
addresses the concept of \emph{firm bankruptcy}, which is the legal
status of a company, in the public registers, that is unable to pay its
debts. A firm is in \emph{default} towards a bank, if it is unable to meet
its legal obligations towards paying a loan. There are specific
quantitative criteria that a bank may use to give a default status to a
company.


In particular, the term "default" should be distinguished from the terms "insolvency", illiquidity and "bankruptcy": \textbf{Illiquidity:} It represents the condition in which a debtor has insufficient cash (or other "liquid"
assets) to pay debts. Illiquidity can be considered a first sign of a problematic situation for the company;  \textbf{Insolvency:} is a legal term meaning a debtor is unable to pay their debts.

But when a company starts to be in a problem situation it becomes very important to be able to foresee the appearance of two particular problematic situations:
\begin{itemize}
\item  \textbf{Firm Default Status:} it is a banking classification of the firm derived from some quantitative criteria that indicate a negative “status” of the company in connection to its banking exposure.
\item  \textbf{Firm Bankruptcy:} it represents the status of bankruptcy
of the company resulting from publishing registers. It is in any case the result of a legal finding.
\end{itemize}

 
\subsection{Definition of \emph{Adjusted Default Status}}
\label{subsec:adjusted-default}

The recent financial crisis has led to a revision and harmonization at
international level of the concept of loan default. In general, default
is the failure to pay interest or principal on a loan or security when
due.

In this paper we consider the classification of \emph{adjusted default
status}, which is a classification that the Italian National Central Bank (Bank of Italy) gives to a company that
has a problematic debt situation towards the entire banking system.

It represents a supervisory concept, whose aim is to extend the default credit status to all the loans of a borrower towards the entire financial system (banks, financial institutions, etc.).

The term refers to the concept of the Basel II international accord of
\emph{default} of customers.
According to this definition, a borrower is defined in default if its
credit exposure has became significantly negative.
In detail, to asses the status of adjusted default, Bank of Italy
considers three types of negative exposures. They are the following, in
decreasing order of severity:

(1) A \emph{bad (performing) loan} is the most negative classification;
(2) an \emph{unlikely to pay} (UTP) loan is a loan for which the bank
has high probability to loose money; (3) A loan is \emph{past due} if it
is not returned after a significant period past the deadline.

Bank of Italy classifies a company in \emph{adjusted default}, or
\emph{adjusted non performing loan} if it has a total
amount of loans belonging to the aforementioned three categories exceeding
certain pre-established proportionality thresholds~\cite{adjusted-default-def}.
 degressive due to the severity of the impaired loan. 
 
Therefore, a firm's adjusted default classification derives from
quantitative criteria and takes into account the company's debt exposure
to the entire banking system.

If a company enters into an adjusted-default status then it is typically
unable to obtain new loans. Furthermore, such companies are multiple
times more likely to bankrupt in the future.

In this paper we  combine two credit dataset in order to predict whether a company will obtain an adjusted default status, although for brevity we may call it just
default. 


\subsection{Firm-Bankruptcy-Prediction Problem}
The concept of the bankruptcy of a business refers to the situation in which business ceases its business being in the condition of not being able to continue the production activity.
This information can be deduced from specific information sources which classify Italian companies on an annual basis based on the condition recorded during each year. The main possible situations are:

\begin{itemize}
\item  Failed
\item  Activate
\item  In insolvency (arrangement with creditors, restructuring, etc.)
\item  In liquidation
\item  Ceased

\end{itemize}


\subsection{Firm Bankruptcy and link with Default status}
In the Italian legal system bankruptcy is a liquidation insolvency
procedure, aimed at satisfying creditors by liquidating the
entrepreneur's assets. We can define “Bankruptcy event” as a target
variable equal to 1 for a company that at time T was identified as
"active" and instead is no longer active in the public register at time
T+1 year. In particular, in the latter case, these company can be
typically in one of the following conditions: bankrupt, inactive, in
liquidation, in insolvency proceedings. %Instead, the target variable is
%equal to 0 (i.e. the firm is not in a %“Bankruptcy status”) if a company
%remains in a “active” status even at time T+1.
%In this paper, we considered a dataset of around 300.000 firms and chose
%initial date T=December 2014. So we taking account companies that at
%such date are in a “good” situation (i.e. not in a “Default status”) and
%were reported as “active” in the public register in order to forecast
%the firm’s condition for the reference date T+1 year.
In order to better explain the relationship between bankruptcy and adjusted bank default status we can consider the following numbers that refer to the period after 2014.
We start our forms observation considering a sample of about 13,000 Italian companies in a "good bank status" in December 2014.
After one year, in December 2015 the financial situation regarding the sample companies was: 10,200 firms have gone classified into a state of
Adjusted Default for the Italian banking system and 2,400 companies are
no longer active having gone bankrupt or being in another similar bad
condition. After two years, in 2016, the bankrupt companies were around
10,000 of which over 2,500 classified in Adjusted Default in 2015.

For instance, out of the
$13K$ companies that were classified in a status of adjusted default
in December of 2015, $2160$ ($16.5\%$) 
were no longer active in 2016, having gone bankrupt or being in another
similar bad condition.
On the other hand, only $2.4\%$ of the companies that were not in adjusted
default status became bankrupt.

%	Default dec 2015 	Bankruptcy 	No Bankruptcy 	No Default dec 2015 	Bankruptcy 	No Bankruptcy
%Dec-15 	13078 	1063 	305403 	293388 	2391 	304075
%Dec-16 	13078 	2160 	304306 	293388 	7169 	299297

%In December 2015 the financial situation regarding our sample of
%companies was: 10,200 firms have gone classified into a state of
%Adjusted Default for the Italian banking system and 2,400 companies are
%no longer active having gone bankrupt or being in another similar bad
%condition. After two years, in 2016, the bankrupt companies were around
%10,000 of which over 2,500 classified in Adjusted Default in 2015.








