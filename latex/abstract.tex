\begin{abstract}



Finding a model to predict the default risk of a firm is a well-known topic over the financial and data science community. Many modern approaches triumph in finding well-performing models to forecast it. Those models often act like a black-box and don't give to financial institutions the fundamental explanations they need for their choices.
This paper aims to find a robust predictive model using a tree-based machine learning algorithm which flanked by a game-theoretic approach can provide sound explanations of the output of the model. Our study uses in combination, for the first time, two large and sparse datasets of credit data from the Italian Central Credit Register of Bank of Italy and from ECB AnaCredit survey that contain information on all Italian companies' past behaviour towards the entire Italian banking system. 
In the end, we show how our model outperforms the current predictions made by institutions and at the same time, provides insights on the reasons that lead to a particular outcome.




%In line with the results of \cite{altman-bankruptcy-17} we find that
%bagging, boosting, and random forest models outperform the others
%techniques, in particular logistic regression. In this sense also our
%research related to default prediction adds to the discussion of the
%continuing debate about superiority of computational methods over
%statistical techniques.


\end{abstract}
